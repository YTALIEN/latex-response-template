\documentclass{article}
\usepackage[utf8]{inputenc}

% template
%% You can change this flag to `False` after finish
\def\IfDraft{True}
%% Template package
\usepackage{responseletter}
%% You can define your custom package in this file
% Using [H] to ban the float of algorithm, figure and so on.
\usepackage{float}

% Figure
\usepackage{graphicx}
\usepackage[]{subfig}

% Table
\usepackage{multirow}

% Math
\usepackage{amsmath}
\usepackage{amssymb}

% URL
\usepackage{url}

% Algorithm
%% Float of algorithm
\usepackage{algorithm}
%% Body of algorithm
\usepackage{algorithmic}
%% Custom setting for algorithmic
\renewcommand{\algorithmicrequire}{\textbf{Input:}} 
\renewcommand{\algorithmicensure}{\textbf{Output:}}

% Autoref
\usepackage[]{hyperref} 
\hypersetup{
  colorlinks=true,
}
%% Custom refname for algorithm
\newcommand{\algorithmautorefname}{Algorithm}
%% Custom refname for subfigure
\newcommand{\subfigureautorefname}{Figure}
%% Custom refname for section
\renewcommand{\sectionautorefname}{Section}
%% Custom refname for subsection
\renewcommand{\subsectionautorefname}{Subsection}
%% Custom refname for step of algorithm
\makeatletter
\newcommand{\ALC@uniqueautorefname}{Line}
\makeatother



% Input the information of paper here
%% The title of the paper
\def\PaperTitle{Title of the paper}
%% ID of paper
\def\PaperId{{JRNL\_YEAR\_NUM}}
%% Revison number
\def\PaperRevision{1}
%% Special issue id
\def\SpecialIssueId{{SI:CONF20XX}}
%% The name of journal or conference
\def\Journal{{Journal of Something}}

%% Names of all authors 
\def\Authors{Name 1, Name 2}
%% Name of author, in the form of `[ID of affiliation]{name}`
\author[1]{Name 1}
\author[2]{Name 2}
%% Information of affiliation, in the form of `[ID of affiliation]{information}`
\affil[1]{University 1, City 1, Country 1}
\affil[2]{University 3, City 2, Country 2}

% Don't show the date
\date{}
\title{\textbf{\PaperTitle}}

% Add your bibtex file here
\addbibresource{ref.bib}

\begin{document}
  
  \maketitle
  
  % Content of the 1st page
  \noindent \textbf{Dear Editors},
  \\[2em]
  \indent We express our gratitude for the time and effort dedicated to the reviewing of our submitted manuscript.
  We worked diligently to address all the concerns raised by the referees.
  Below we provide our detailed response to their comments.
  %We have highlighted the main changes in the paper by colouring the modified text and adding a side note in which the addressed remark is referenced.
  We hope that the applied revisions are to the satisfaction of the editors.
  \\[2em]
  Kind regards,
  \begin{flushright}
    \Authors
  \end{flushright}
  \vfill
  \section*{Manuscript information}
  \begin{description}
    \item[Number:] \PaperId
    \item[Title:] ``\PaperTitle''
    \item[Authors:] \Authors
    \item[Revision:] \PaperRevision
    \item[Submitted to:] \Journal
    \item[Special issue:] \SpecialIssueId
  \end{description}
  \pagebreak
  
  {
    \hypersetup{linkcolor=black}
    \tableofcontents
    \pagebreak
  }

  \begin{Editor}[Associate Editor]
  \begin{CommentSummary}
    Hope the reviewers' comments would be useful for your research.
  \end{CommentSummary}
  
  \begin{Response}
    Thank you very much.
    We have revised the manuscript by taking reviewers’ comments and suggestions into consideration to enhance the paper quality.
    The revisions are marked in {\color{blue}blue} color in the revised manuscript.
  \end{Response}
\end{Editor}
  \begin{Reviewer}

  \begin{CommentSummary}
    The paper puts forward a new methodology for XXX.
    The concepts are clearly described and the algorithmic flows are presented carefully.
    Experiments are carried out in comparison to state-of-the-art methods on a set of benchmark functions.
  \end{CommentSummary}
  \begin{Response}
    Thank you very much for your valuable comments on our paper and work.
    We have revised the manuscript by taking your following comments and suggestions into consideration to enhance the paper quality.
    The revisions are marked in {\color{blue}blue} color in the revised manuscript. 
  \end{Response}

  \begin{ReviewerComment}
    The writing is generally good but could be further improved.
  \end{ReviewerComment}
  
  \begin{Response}
    Thank you for your valuable comment.
    We have further polished up the language presentations.
    We hope the revised paper will be more clear and accurate on expressions.
  \end{Response}

  \begin{ReviewerComment}
    The process of XXX is not clear.
  \end{ReviewerComment}
  
  \begin{Response}
    Thank you for your valuable comment. 
    In the revised paper, we have added some XXX.
    The revised description are shown in the Section XXX-X. 
  \end{Response}

  \begin{ReviewerComment}
    There are some typos and informal notations, e.g., the decision vector is a vector and thus it should be given as $\mathbf{x}$.
  \end{ReviewerComment}
  \begin{Response}
    Thank you very much for pointing out this issue.
    We have changed the notations according to your suggestion and double-checked the manuscript to avoid similar problems. 
  \end{Response}


  \begin{ReviewerComment}
    The format of the references should be unified, e.g., [8]  and [38].
  \end{ReviewerComment}

  \begin{Response}
    Thank you very much for pointing out this issue.
    We have carefully formatted all the references in the revised manuscript.
    The above three reference formats are presented as follows:

    \begin{enumerate}
      \item[8] \fullcite{kahn1962topological}

      \item[38] \fullcite{he2016deep}
    \end{enumerate}
  \end{Response}

\end{Reviewer}

\end{document}
